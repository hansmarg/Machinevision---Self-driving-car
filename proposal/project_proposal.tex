\documentclass{article}
\usepackage[utf8]{inputenc}

\title{Project proposal}
\author{Forat Al-Hellali, Hans-Marthin Granlund }
\date{April 2018}

%\usepackage{natbib}
%\usepackage{graphicx}

\begin{document}

\maketitle

\section{Introduction}
The project will be the data-collecting part of an autonomous driving car. The program will analyze a dashcam video and will contain the following:
\begin{itemize}
    \item Street detection
        \begin{itemize}
            \item road segmentation
            \item lane slope detection
        \end{itemize}
    \item Object detection
        \begin{itemize}
            \item other cars
            \item signs
            \item pedestrians
        \end{itemize}
    \item Calculate distance
    \item Calculate relative speed
    \item 3D mapping of the world
\end{itemize}

\section{Plan}
\subsection{Street Detection}
\subsubsection{Segmentation}
First we would down sample and filter the image, by using the appropriate the appropriate filters like: Laplace, Gaussian, or Difference of Gaussian.

Then we would experiment with different segmentation algorithms to highlight the road.

\subsubsection{Lane detection}
Use the Sobel filter to highlight lines in the picture and then run RANSAC to find the lane lines.

\subsection{Object Detection}
Train a neural network to recognize cars using the training data available from open source machine learning training databases.
If this works out, then we would train the network to recognize pedestrians and street signs.

\subsection{Calculate Distance}
We can use the width of the lane lines (closest to the car) as a constant to calculate distance since they are mostly parallel.


\subsection{Calculate Relative Speed}
$$relative\_velocity = \frac{new\_estimated\_object\_distance - old\_estimated\_object\_distance}{time}$$
Given time we will experiment with using histogram of gradients to calculate the velocity.

\subsection{3D reconstruction}
Use most of the information we gain to reconstruct a very simple 3d figure of the environment. Like the one shown in the picture.

\section{Questions}
\begin{itemize}
\item Do you have a database for better a trainingset than ours?
\item Any suggestions of how we would use the histogram of gradients to calculate the relative speed?
\end{itemize}

\end{document}
